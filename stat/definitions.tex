% Created 2021-12-22 Wed 09:50
% Intended LaTeX compiler: pdflatex
\documentclass[11pt]{article}
\usepackage[utf8]{inputenc}
\usepackage[T1]{fontenc}
\usepackage{graphicx}
\usepackage{grffile}
\usepackage{longtable}
\usepackage{wrapfig}
\usepackage{rotating}
\usepackage[normalem]{ulem}
\usepackage{amsmath}
\usepackage{textcomp}
\usepackage{amssymb}
\usepackage{capt-of}
\usepackage{hyperref}
\author{Sreejith Sreekumar}
\date{\today}
\title{Stat Definitions}
\hypersetup{
 pdfauthor={Sreejith Sreekumar},
 pdftitle={Stat Definitions},
 pdfkeywords={},
 pdfsubject={},
 pdfcreator={Emacs 27.2 (Org mode 9.4.4)}, 
 pdflang={English}}
\begin{document}

\maketitle
\tableofcontents


\section{\underline{Confidence Interval}}
\label{sec:org87b7a6f}
\begin{itemize}
\item A confidence interval gives the PROBABILITY that our true value lies within the range of values. Bigger interval = higher probability
\end{itemize}

\section{\underline{Probability}}
\label{sec:org70dd510}
\begin{itemize}
\item Area under an interval of a distribution curve.
\end{itemize}

\section{\underline{Likelihood} vs Probability}
\label{sec:orgfe0104d}

\begin{itemize}
\item \href{https://stats.stackexchange.com/a/183885/84189}{Answer from Cross Validated}
Likelihood: What is the best values of the parameters so that the data that we observed follows a <some> distribution?
Probability: Assuming that the data comes from a certain distribution what is the chance. Probability is the area under the PDF curve
\end{itemize}

\section{\underline{Percentage of normal distribution lies within 1 std of mean? 2, 3 std?}}
\label{sec:org18540c4}
\begin{itemize}
\item 68\%, 95\%, 99.7\%
\end{itemize}

\section{\underline{SGD Update Rule}}
\label{sec:orgbcec943}

$$\theta = \theta - \alpha \Delta J(\theta)$$
$$(Current\ \theta \ vector) - learning\ rate * (Gradient\ of\ Slope)$$

\section{\underline{Probability and Statistics}}
\label{sec:org778496e}

The problems considered by probability and statistics are inverse to each other.
In probability theory we consider some underlying process which has some randomness or uncertainty modeled by random variables, and we figure out what happens.
In statistics we observe something that has happened, and try to figure out what underlying process would explain those observations.

\section{\underline{Law of Large Numbers}}
\label{sec:orga05d1a4}

If you repeat an experiment independently a large number of times and average the result, what you obtain should be close to the expected value

\section{\underline{Central Limit Theorem}}
\label{sec:org750af85}

\section{\underline{Type I and Type II Error}}
\label{sec:org849d720}

Type 1: False Positive, Type 2: False Negative

\section{\underline{Inverse Document Frequency}}
\label{sec:org7c8033c}

$$idf = log \frac{|D|}{d : ti \in d}$$
where | D | is the number of documents in our corpus, and | \{d : ti \(\in\) d\} | is the number of documents in which the term appears.

\section{\underline{Kolmogorov - Smirnov Test}}
\label{sec:org49a3dc2}

Tests whether sample fits a distribution well.

\section{\underline{Confidence Interval}}
\label{sec:orgb476252}

There are 100 products and 25 of them are bad. What is the confidence interval?

p = 25/100 = 0.25

CI = 0.25 +/- 1.96 sqrt( (0.25(1-0.25)) * 100)
CI = p +/- Z * sqrt(variance of binom dist)

CI = (16.5,33.5)

95\% confidence = plus or minus 1.96 STDEV

\section{\underline{Isolation Forest}}
\label{sec:orge7468f1}

Creates splits like a random forest, but find out how difficult is it to isolate the path to split an instance.
Answer: \href{https://www.quora.com/What-is-the-difference-between-random-forest-and-isolation-forest}{Quora}
\end{document}
